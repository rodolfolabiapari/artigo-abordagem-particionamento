%!TEX root = ../main.tex
% !TeX encoding = UTF-8
\subsection{Trabalhos Relacionados}  \label{chap:relacionados}
    % Embedded
    %O desenvolvimento com foco em SE ou microcontroladores já é pesquisado amplamente como os trabalhos de \cite{Ernst1993, Gupta1995, Hardt1995, Gajski1994, Bolsens1997}, publicados na década de 90.
    
    Em \cite{Mei2000} além do particionamento, os autores apresentam uma abordagem de escalonamento para SE dinamicamente reconfigurável (DRESs, do inglês \textit{dynamically reconfigurable embedded systems}). % no qual possuem como projeto um processador de propósito geral junto com um FPGA sendo este reconfigurável em tempo de execução para reduzir custos.
    Neste trabalho os autores fornecem análises dos tempos de configuração e reconfiguração parcial do FPGA e mostram que o algoritmo proposto resolve o problema de particionamento e escalonamento dos DRES.
    
    Em \cite{Arato2003} os autores descrevem algumas versões do problema de particionamento para sistemas de tempo real e custo restringido, provando que são problemas $ \mathcal{NP} $-difícil.
    Adicionalmente, apresentam uma abordagem com programação linear inteira, resolvendo o problema de forma otimizada, e uma abordagem utilizando algoritmo genético, na qual encontram-se soluções próximas ao ótimo global.
    
    O trabalho em \cite{Mann2007} descreve uma primeira tentativa para um algoritmo exato para o problema de particionamento.
    Utiliza um esquema no qual implementa-se a estratégia \textit{branch-and-bound} como um \textit{framework}.
    %Em sua implementação, realizaram várias investigações para incrementar a eficiência do algoritmo, incluindo várias técnicas sendo elas: \textit{lower bounds based on LP-relaxation}, uma mecânica de inferência customizada, condições não-triviais necessárias baseadas num algoritmo \textit{minimum-cut}, e diferentes heurísticas com passos pré-otimizados.
    %Este também pode ser generalizado a fim de incluir mais de uma restrição, permitindo o \designer\ prescrever quais itens devem estar em qual nível de projeto.
    Eles demonstram que problemas de particionamento altamente complexos podem ser resolvidos em tempo razoável.
    %Citam ao final que o resultado obtido é em entorno de dez minutos mais rápido que algoritmos exatos anteriores baseados em programação linear inteira para os testes realizados.
    
    Pesquisas mais recentes, como a de \cite{BenHajHassine2017} procuram aplicar otimizações sobre o tempo de execução e gasto energético para processadores baseados em produtos embarcados, por meio de particionamento.
    %Propõem alcançar um particionamento de grafos à procurar o melhor conjunto da relação energia e tempo de execução.
    Comparado com outras heurísticas, o algoritmo mostra-se ser mais adequado para aplicações que necessitam do equilíbrio no \textit{tradeoff}.
    
    Trabalhos como o de \cite{Trindade2016} utilizam algoritmos genéticos para solucionar o problema de particionamento em SE.
    Propõem novas abordagens usando técnicas de verificação baseadas nas teorias de módulo de satisfação (SMT, do inglês \textit{satisfiability modulo theories}).
    %Apresentam um exemplo de particionamento, modelam e solucionam-o usando três diferentes técnicas sendo a principal ideia é aplicar mo método de verificação SMT ao particionamento \hs, e por fim, comparar os resultados com técnicas de otimizações tradicionais como ILP e GA.
    
    
    %concluindo tudo que foi dito
    Os trabalhos citados buscam o estudo do desempenho e \design\ de SE no geral por meio de particionamento, mas nenhum com foco em \wearables.
    
    
    Em \cite{Jozwiak2017} os autores apresentam uma extensa revisão da literatura considerando vários aspectos de um SE, bem como suas tecnologias de \design,\ com foco em sistemas móveis modernos incluindo \wearables.
    %Cita-se dois paradigmas de desenvolvimento para SE sobre sistema multi-processados heterogêneos, sendo eles o de sistemas \textit{life-inspired} e sistemas \textit{quality-driven}.
   % O \textit{life-inspired} (inspirado pela vida) especifica princípios básicos, características e organizações funcionais e estruturais de um SEe por meio da analogia à vida de um organismo inteligente, além de soluções de mecanismos e arquiteturas de sistemas para implementar tais princípios.
    %Já o \textit{quality-driven} (orientado pela qualidade) é o \design\ de dispositivos que necessitam satisfazer as exigências de tempo real, baixo consumo de energia, entre outros e assim, especifica qual a nova qualidade do sistema a ser requerida e como esta meta é obtida.
    %De forma a facilitar a compreensão, \cite{Jozwiak2017} define qualidade de uma solução sistêmica proposto como o total de sua eficácia e eficiência na resolução do problema real.
    %Eficácia entende-se como o grau em que uma solução atinge seus objetivos e a eficiência o grau em que uma solução usa recursos para realizar seus objetivos e juntas determinam o grau de excelência.
    %Elas são expressas em termos de parâmetros mensuráveis, o que é necessário para implementar o design \textit{quality-driven}.
    %Entretanto, é descrito ao final que, enquanto \designers\ aprenderam bastante na construção de plataformas de \hardware\ heterogêneos altamente paralelos, os métodos e ferramentas automatizadas para a sua programação e o paralelismo do algoritmo, bem como o \codesign\ coerente da arquitetura \hs\ ainda são atrasados perante à tecnologia.
    % wearable fpga
    Além deste, é possível ver em \cite{Plessl2003, Ahola2007, Abdelhedi2016, Narumi2016, Lee2015} trabalhos que estudam \wearables\ junto de FPGA, mas nenhum deles utilizam a técnica de particionamento como meio para seu \design.
    
    % conclusao
    Este trabalho portanto consiste na análise do problema de particionamento \hs,\ com foco em \design\ de sistemas \wearables\ em plataforma FPGA.
