%!TEX root = ../main.tex
% !TeX encoding = UTF-8
\section{Conclusões} \label{chap:conclu}
    %projeto de sistemas
    A demanda por curto tempo para disponibilidade ao mercado, somado ao fato dos produtos exigirem propriedades de corretude, rapidez, confiabilidade e preço acessível representam um desafio para projetistas de sistemas embarcados em geral.
    %utiliza o particionamento para o problema de desempenho
    Com o desenvolvimento de sistemas embarcados cada vez mais complexos, o particionamento \hs\ tornou-se um problema de otimização em \codesign\ de sistemas.
    %wearable
    Como dispositivos \wearables\ também demandam um alto desempenho e/ou baixo consumo de energia sem apresentar desequilíbrio em confiabilidade e segurança entre outros, aplicou-se o particionamento sobre essa classe de sistemas, com foco em tais melhorias.
    
    %proposta
    A proposta da pesquisa constituiu-se na busca pelo aprimoramento de desempenho de dispositivos computacionais \wearables,\ utilizando o particionamento como meio, visando gasto energético e de recursos limitados de plataforma FPGA.
    
    %comentando os testes
    Para a avaliação, realizou-se particionamento de quatro algoritmos candidatos (Estatístico \A$_{Es}$, Lagrange \A$_{La}$, Números Primos  \A$_{NP}$ e Processamento de Risco \A$_{Ri}$) de um projeto de capacete de segurança para ciclistas, variando cada teste em quantidade de sensores e também o \buffer\ de operação.
    
    % comentando os resultados
    Três dos quatro sistemas (Lagrange \Ss$_{La}$, Números Primos \Ss$_{NP}$ e Risco \Ss$_{Ri}$) obtiveram sucesso na busca por desempenho apenas pelo processo de particionamento, aumentando no mínimo $9,6\%$ seus desempenhos, utilizando o valor máximo de $ 5,5\% $ de recursos e $ 5,4\% $ de energia do \hardware\ reconfigurável.
    %
    O sistema Estatístico \Ss$_{Es}$ em $45,5\%$ dos testes obteve maior desempenho \software\ comparado com \hardware.
    Esse resultado já era esperado, já que seu código exige bastante comunicação entre \hs, afetando seu desempenho.

    %trabalhos futuros
    %har e softprocessadores
    Para futuros trabalhos seria possível realizar a comparação do sistema \wearable\ particionado variando, também a arquitetura de sintetização, ou seja, testes de desempenho entre \textit{soft} e \textit{hard-}processadores, ambos utilizando FPGA.
    
    % fpga e prototipações
    %Comparações entre o uso de plataformas FPGA e plataformas de prototipações também seriam viáveis, ainda com foco em busca em otimizações em desempenho e eficiência energética sobre \wearables.
    
    %mais alguma sugestão de trabalhos?
    %E também a adição do parâmetro otimização em \hardware,\ verificando o percentual de ganho ao utilizar as técnicas de otimizações em nível de \hardware\ existentes para HLS.
    
    
    % use section* for acknowledgment
    \section*{Agradecimentos}
        Agradecemos à Universidade Federal de Ouro Preto, ao CNPq, CAPES e à FAPEMIG pelo subsídio dessa pesquisa.        

   % conference papers do not normally have an appendix
   %\section*{Apêndice}    
   %    \hphantom{a}
